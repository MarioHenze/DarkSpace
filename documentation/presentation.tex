\documentclass{beamer}

\usepackage[ngerman]{babel}

\usepackage{
adjustbox,
tikz,
tikzsymbols,
pgfplots,
physics,
siunitx,
xcolor
}

\usepackage{tikz-3dplot-circleofsphere}

\pgfplotsset{compat=newest}

\title{Virtual Reality:\\Spielen ohne Grafik?}

\date{09.07.2019}

\begin{document}

\begin{frame}
\titlepage
\end{frame}

\begin{frame}
\frametitle{Spielkonzeption}

\begin{adjustbox}{max totalsize={.9\textwidth}{.7\textheight}, center}
\begin{tikzpicture}
%ground
\draw (-0.2, -.2) -- (2.2, -.2);
\draw (2.8, -.2) -- (5.2, -.2);
\draw (5.8, -.2) -- (8.2, -.2);

%cells
\draw (0, 0) rectangle (2, 2);
\fill[gray](3, 0) rectangle (5, 2);
\draw (6, 0) rectangle (8, 2);

%guy
\node at (1, 1) {\Strichmaxerl[2]};
\end{tikzpicture}
\end{adjustbox}

\begin{itemize}
	\item drei feste Positionen
	\item Hindernisse fliegen von hinten auf Spieler zu
	\item Figur bleibt ohne Eingabe in mittlerer Position
	\item Interaktion durch Ausweichen nach links oder rechts
\end{itemize}

\end{frame}

\begin{frame}
\frametitle{Psychoakustischer Beschreibungsraum}

\begin{columns}
	\column{0.46\linewidth}
	\begin{itemize}
		\item primär
		\begin{itemize}
			\item Frequenzmodulation
			\item Amplitudenmodulation
			\item binaurales Hören
		\end{itemize}

		\item sekundär
		\begin{itemize}
			\item Schwankungsstärke
			\item Rauigkeit
			\item Schärfe
			\item Tonhaltigkeit
		\end{itemize}
	\end{itemize}
	\column{0.54\linewidth}
	\def\r{1.5}
\tdplotsetmaincoords{60}{125}
\begin{tikzpicture}[tdplot_main_coords]
	\begin{scope}[thin, black!30]
		\draw[->] (-1.3 * \r,0,0) -- (1.3 * \r,0,0) node[anchor=north east]{$x$};
		\draw[->] (0,-1.3 * \r,0) -- (0,1.3 * \r,0) node[anchor=north west]{$y$};
		\draw[->] (0,0,-1.3 * \r) -- (0,0,1.3 * \r) node[anchor=south]{$z$};
	\end{scope}

	\tdplotCsDrawLatCircle{\r}{0}
	\tdplotCsDrawLonCircle{\r}{0}
	\tdplotCsDrawCircle{\r}{90}{90}{0}

	\draw ({sin(45) * -\r}, {sin(45) * \r}, 0) --
		({(sin(45) * -\r) * 1.2}, {(sin(45) * \r) * 1.2}, 0)
		node[anchor=west] {Horizontalebene};
	\draw ({sin(45) * -\r}, 0, {sin(45) * \r}) --
		({(sin(45) * -\r) * 1.2}, 0, {(sin(45) * \r) * 1.2})
		node[anchor=west] {Frontalebene};
	\draw (0, {sin(180 + 45) * -\r}, {sin(180 + 45) * \r}) --
		(0, {(sin(180 + 45) * -\r) * 1.2}, {(sin(180 + 45) * \r) * 1.2})
		node[anchor=west] {Medianebene};
\end{tikzpicture}
\end{columns}
\end{frame}

\begin{frame}
\frametitle{akustische Kodierung}
\begin{itemize}
	\item Auflösung binauralen Hörens für kontinuierliche Merkmale zu gering
	\begin{itemize}
		\item Ausschließlich qualitative Merkmale möglich (z.B. mittig oder rechts)
	\end{itemize}
	\item Sekundäre Merkmale werden durch primäre erzeugt
	\item Nur Frequenz und Amplitude für Position und Entfernung geeignet
\end{itemize}
\end{frame}



\begin{frame}
	\frametitle{Fazit}
	\begin{itemize}
		\item Informationsübermittlung stößt schnell an Grenzen
		\item komplexe Szenen führen zu aliasing zwischen primären und sekundären Merkmalen
		\item 
	\end{itemize}
\end{frame}

\end{document}
