\documentclass[a4paper, 11pt]{scrartcl}

\usepackage[utf8]{inputenc}

\usepackage[ngerman, english]{babel}

\usepackage{tikz,tikz-3dplot-circleofsphere}

\begin{document}

\section{Psychoakustische Grundlagen}
\label{sec:psychoakustische_grundlagen}

\subsection{Frequenz}
\label{sub:frequenz}

Die Frequenz ist ein Maß wie oft sich der Schalldruck pro Sekunde ändert und
wird in Hertz gemessen. Sie korreliert mit der wahrgenommen Tonhöhe eines
Audiosignals.

\subsection{Amplitude}
\label{sub:Amplitude}

Die Amplitude gibt die absolute Differenz zwischen Maximal- und Normaldruck an
und korreliert mit der wahrgenommen Lautstärke eines Audiosignals.

\subsection{Binaurales Hören}
\label{sub:binaurales_hoeren}

\begin{figure}
\centering
\def\r{1.5}
\tdplotsetmaincoords{60}{125}
\begin{tikzpicture}[tdplot_main_coords]
	\begin{scope}[thin, black!30]
		\draw[->] (-1.3 * \r,0,0) -- (1.3 * \r,0,0) node[anchor=north east]{$x$};
		\draw[->] (0,-1.3 * \r,0) -- (0,1.3 * \r,0) node[anchor=north west]{$y$};
		\draw[->] (0,0,-1.3 * \r) -- (0,0,1.3 * \r) node[anchor=south]{$z$};
	\end{scope}

	\tdplotCsDrawLatCircle{\r}{0}
	\tdplotCsDrawLonCircle{\r}{0}
	\tdplotCsDrawCircle{\r}{90}{90}{0}

	\draw ({sin(45) * -\r}, {sin(45) * \r}, 0) --
		({(sin(45) * -\r) * 1.2}, {(sin(45) * \r) * 1.2}, 0)
		node[anchor=west] {Horizontalebene};
	\draw ({sin(45) * -\r}, 0, {sin(45) * \r}) --
		({(sin(45) * -\r) * 1.2}, 0, {(sin(45) * \r) * 1.2})
		node[anchor=west] {Frontalebene};
	\draw (0, {sin(180 + 45) * -\r}, {sin(180 + 45) * \r}) --
		(0, {(sin(180 + 45) * -\r) * 1.2}, {(sin(180 + 45) * \r) * 1.2})
		node[anchor=west] {Medianebene};
\end{tikzpicture}

\caption{Die drei Ebenen des binaural Hörens unter der Voraussetzung, dass der 
Blick in y-Richtung verläuft und die Ohren auf der x-Achse liegen.}

\label{fig:spatilisation}
\end{figure}

Um den Effekt einer richtungsabhängigen Wahrnehmung eines Schallereignisses zu
erreichen, können Laufzeit- und Pegeldifferenzen genutzt werden. Aus dem
Unterschied zwischen beiden Ohren kann das Gehirn das Hörereignis auf der
Horizontalebene lokalisieren. Durch Nutzung der blauertschen Bänder ist es
außerdem möglich auf der Medianebene zu lokalisieren.

\subsection{Schwankungsstärke}
\label{sub:schwankungsstaerke}

Die Schwankungsstärke gibt die empfundene Schwankung in der Lautstärke eines
Audiosignals an und wird in vacil gemessen.

\subsection{Rauigkeit}
\label{sub:rauigkeit}

Die Rauigkeit misst die empfundene Rauigkeit im Timbre eines Audiosignals und
wird in asper angegeben.

\subsection{Schärfe}
\label{sub:schaerfe}

Die Schärfe eines Audiosignals bezeichnet das wahrgenomme Verhältnis von hohen
zu tiefen Frequenzen und wird in acum gemessen.

\subsection{Tonhaltigkeit}
\label{sub:tonhaltigkeit}

Die Tonhaltigkeit ist ein Maß um die Wahrnehmbarkeit von Einzeltönen im Spektrum
eines Audiosignals anzugeben und wird in Aures gemessen.

\section{Audiomodell}
\label{sec:audiomodell}

Damit der Spieler mit der Spielwelt interagieren kann, muss ein bidirektionaler
Austausch zwischen Spieler und Spiel entstehen. Da im Rahmen der Projektarbeit
die Bedingung gesetzt wurde, Menschen mit Sehbehinderung als Zielgruppe zu
wählen, fällt der visuelle Austausch von Information aus. Die Herausforderung
besteht somit, relevante Information der Spielmechanik zu bestimmen und diese in
geeigneter Weise auf akustische Cues zu transformieren. Die im
Kapitel~\ref{sec:psychoakustische_grundlagen} vorgestellten Parameter bilden
hierfür einen Beschreibungsraum von Geräuschen, welche die Information der
Spielmechanik an den Spieler mitteilen.

Bei der direkten Übersetzung von räumlicher Position auf binaurales Hören
mithilfe von Pegel- und Laufzeitunterschieden fällt die geringe Auflösung der
Wahrnehmung auf. So sind ausschließlich qualitative Urteile wie „mittig“,
„rechts“ oder „schräg rechts“ möglich. Folglich können räumliche Informationen
aus der Spielmechanik nicht direkt durch binaurales Hören kodiert werden,
sondern müssen durch Parameter mit höherer Sensibilität kommuniziert werden.
Durch die multimodale Wahrnehmung können Pegel- und Laufzeitunterschiede jedoch
als unterstüzender Akustik-Cue verwendet werden. Das heißt die räumlichen
Informationen werden sowohl mit distinktem Parameter als auch mit binauralem
Hören kodiert. Durch diese Redundanz sollten räumliche Information deutlich
präziser und schneller wahrgenommen werden können.

\section{Spielmechanik}
\label{sec:spielmechanik}



\end{document}